% ---
% Capitulo de fundamentação teórica
% ---
\chapter{Fundamentação teórica}

Para propor uma solução tecnológica para auxílio nos cuidados de idosos, é preciso conhecer e entender quais as demandas e oportunidades relacionadas à gerontologia, e quais as ferramentas computacionais disponíveis para atender a essas demandas. Por este trabalho ser parte de uma especialização em desenvolvimento para dispositivos móveis, será tratado mais detalhadamente os aspectos da computação móvel. No entanto, passaremos superficialmente por assuntos relacionados ao envelhecimento humano, a fim de identificar o que deve ser oferecido pelo assistente tecnológico, quais os aspectos a serem considerados e como introduzir estas soluções no cotidiano do idoso e seus cuidadores e responsáveis.

Este capítulo, portanto, apresenta uma breve investigação sobre as principais síndromes geriátricas, a serem melhor exploradas posteriormente, no estudo de viabilidade do projeto. Em seguida são apresentados os principais conceitos, terminologias, modelos e padrões de projetos para dispositivos móveis utilizados no desenvolvimento deste trabalho.

\section{Principais síndromes geriátricas}

\section{Dispositivos móveis}

\subsection{Android}

\subsection{Material Design}

\section{Modelo Canvas}

\section{Matriz de valor}


